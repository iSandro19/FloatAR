\documentclass[a4paper, openright, 12pt]{article}
\usepackage[utf8]{inputenc}
\usepackage{graphicx} 
\usepackage{subfigure}
\usepackage[mathscr]{eucal}
\usepackage{titling}
\usepackage{float}
\usepackage{amsmath}
\usepackage{afterpage}
\usepackage{vmargin}
\usepackage[spanish]{babel}
\usepackage{eurosym} 
\usepackage{multirow} 
\usepackage{cite}
\usepackage{url}

\setpapersize{A4} %  DIN A4
\setmargins{3cm}  % margen izquierdo
{3.5cm}           % margen superior
{15cm}            % anchura del texto
{22.5cm}          % altura del texto
{10pt}            % altura de los encabezados
{1cm}             % espacio entre el texto y los encabezados
{0pt}             % altura del pie de página
{2cm}             % espacio entre el texto y el pie de página

\begin{document}

\begin{titlepage}

\begin{center}
\vspace*{-1in}
\begin{figure}[htb]
\begin{center}
\includegraphics[width=8cm]{udc.eps}
\end{center}
\end{figure}

\vspace*{1in}
PROGRAMACIÓN DE SISTEMAS 22/23 Q2\\
\begin{figure}[htb]
\begin{center}
\includegraphics[width=8cm]{icono.png}
\end{center}
\end{figure}
\begin{Large}
\textbf{FloatAR} \\
\end{Large}

\vspace*{1in}

\begin{large}
\raggedleft
\textbf{Autores:}\\
{Antonio Vila Leis - antonio.vila@udc.es\\Hugo Sanjiao Varela - h.sanjiao@udc.es\\Breogán Fernández Moreira - breogan.fernandez@udc.es\\Óscar Alejandro Manteiga Seoane - oscar.manteiga@udc.es\\}

\vspace{1in}

\textbf{Fecha:}\textit{ A Coruña, 2 Junio 2023}\\
\end{large}

\end{center}
\end{titlepage} 

\newpage

\addtocontents{toc}{\hspace{-7.5mm} \textbf{Capítulos}}
\addtocontents{toc}{\hfill \textbf{Página} \par}
\addtocontents{toc}{\vspace{-2mm} \hspace{-7.5mm} \hrule \par}

\pagenumbering{empty}

\tableofcontents

\vspace{5cm}

\begin{flushright}
\begin{table}[hbtp]
\begin{center}

\caption{Tabla de versiones.}
\label{tabla:versiones}
\small
\vspace{1ex}

\begin{tabular}{|c|c|l|}
\hline
Versión & Fecha & Autor \\
\hline \hline
1 & 16 de marzo de 2023 & Antonio | Hugo | Breogán | Óscar\\ \hline \hline
2 & 18 de abril de 2023 & Antonio | Hugo | Breogán | Óscar\\ \hline \hline
3 & 9 de mayo de 2023 & Antonio | Hugo | Breogán | Óscar \\ \hline
4 & 12 de mayo de 2023 & Antonio | Hugo | Breogán | Óscar \\ \hline
5 & 2 de junio de 2023 & Antonio | Hugo | Breogán | Óscar \\ \hline
\end{tabular}

\end{center}
\end{table}
\end{flushright}

\pagenumbering{arabic}
\newpage

%%%%%%%
%%%%%%%
\section{Introducción}\label{cap.introduccion}
La aplicación FloatAR es una aplicación de realidad aumentada del juego "Hundir la flota" para dispositivos Android. Esta aplicación utiliza la API ARCore y se desarrollará para un trabajo de clase de la asignatura Programación de Sistemas en un grupo de 4 personas. El juego podrá ser utilizado de forma estándar o con el uso del modo de realidad aumentada.
%%
\subsection{Objetivos}
El objetivo principal de FloatAR es proporcionar una experiencia de juego más inmersiva para los usuarios utilizando la tecnología de realidad aumentada. Esta tecnología no está muy expandida en las aplicaciones de Android, pero aun menos en los videojuegos, por lo que otro objetivo implícito podría ser la investigación de esta tecnología. Los objetivos secundarios incluyen:
\begin{itemize}
    \item Desarrollar de forma convencional el juego de "Hundir la flota" en modo local.
    \item Implementar el modo para 2 jugadores vía Internet.
    \item Integrar un sistema de puntuación y registro de puntuación para los usuarios.
    \item Añadir el espacio de juego al mundo real con la realidad aumentada utilizando la API ARCore \cite{ARCore}.
\end{itemize}
%%
\subsection{Motivación}
La tecnología de realidad aumentada ha ganado popularidad en los últimos años y se está utilizando cada vez más en juegos y aplicaciones. El uso de la realidad aumentada en juegos como hundir la flota puede proporcionar una experiencia de juego más inmersiva y emocionante para los usuarios. A mayores, la investigación e implementación de esta tecnología en juegos es un apartado que nos produce gran interés, ya que no está muy extendida y creemos que será muy usada en el futuro. La aplicación FloatAR también puede ser aplicable en otros juegos que requieren que ambos jugadores (o más de 2) no puedan ver lo que ven los demás. Esto hace que en un futuro próximo podamos expandir el juego si es necesario.

%%
\subsection{Trabajo relacionado}
Hay otras aplicaciones de realidad aumentada para juegos, como Pokémon Go (Niantic) \cite{PokemonGO}, la aplicación más exitosa que implementa realidad aumentada (y una de las más exitosas a nivel de aplicación general) que utilizan esta tecnología para crear una experiencia de juego más inmersiva. Otros ejemplos son Angry Birds AR: Isle of Pigs (Rovio Enterteinment) \cite{AngryBirds}, Jurassic World Alive (Ludia Inc.) \cite{JurassicWorld}, Harry Potter: Wizards Unite (Niantic) \cite{HarryPotter}, Minecraft Earth (Mojang) \cite{MinecraftEarth}... Todas ellas implementan la RA, pero de formas muy distintas:
\begin{itemize}
    \item Pokémon GO, Jurassic World Alive y Harry Potter: Wizards Unite la usan para mostrar sus Pokémons, dinosaurios o entidades fantásticas.
    \item Minecraft Earth la usa para mostrar su mundo en nuestra habitaciones o espacios de la vida real a través de la cámara.
    \item Angry Birds AR: Isle of Pigs y nuestra aplicación FloatAR la usan para mostrar el espacio de juego en la vida real para ver como se desarrolla el evento.
\end{itemize}

%%%%%%%
%%%%%%%
\section{Análisis de requisitos}
Antes de comenzar el desarrollo de la aplicación FloatAR, se debe realizar un análisis preliminar de los requisitos. Esto ayudará a definir las funcionalidades clave de la aplicación y establecer prioridades.
\begin{itemize}
    \item Jugar de forma local contra la IA sin realidad aumentada.
    \item Jugar contra otro jugador sin realidad aumentada.
    \item Interfaz de usuario intuitiva y simple con sistema de puntuación.
    \item Modo de realidad aumentada.
\end{itemize}
%%
\subsection{Funcionalidades}
\begin{enumerate}
    \item Tablero de juego sin ser en realidad aumentada.
    \item Colocación de barcos en el tablero con sistema de colisiones.
    \item Menú de ajustes, de información y de ayuda.
    \item Juego local contra la IA, con turnos, detección de victoria/derrota...
    \item Juego online contra otro jugador con las mismas características.
    \item Posibilidad de jugar a algún modo en realidad aumentada.
\end{enumerate}
%%
\subsection{Prioridades}
\begin{enumerate}
    \item Crear un tablero sin ser en realidad aumentada (Core).
        \begin{enumerate}
            \item Permitir al usuario colocar sus barcos en el tablero virtual (Core).
            \item Implementar un sistema de detección de colisiones para asegurar que los barcos no se superpongan (Core).
        \end{enumerate}
    \item Crear pantalla de ajustes, ayuda e información sobre la aplicación (Core).
    \item Implementar sistema de juego local (Core).
        \begin{enumerate}
            \item Implementar un sistema de juego por turnos contra una IA (Core).
            \item Implementar un sistema de detección de victoria y derrotas (Core).
         \end{enumerate}
    \item Implementar sistema de juego online (Core).
        \begin{enumerate}
            \item Implementar el sistema de comunicación entre jugadores para sincronizar el juego y las puntuaciones/eventos (Core).
            \item Implementar un sistema de juego por turnos contra otro jugador (Core).
        \end{enumerate}
    \item Desarrollar la realidad aumentada (Core).
        \begin{enumerate}
            \item Desarrollar la parte de realidad aumentada.
        \end{enumerate}
    \item Mostrar la puntuación y el estado del juego en tiempo real (Accesorio).
    \item Implementar un sistema de animación para ciertas acciones (Accesorio).
\end{enumerate}

%%%%%%%
%%%%%%%
\section{Planificación inicial}
En esta sección se establecerá una planificación muy simple y básica del proyecto, definiendo las iteraciones, responsabilidades, hitos y entregables.
%%
\subsection{Iteraciones}
Consideramos 12 semanas para la implementación de las funcionalidades del juego:
\begin{itemize}
    \item Inicio del proyecto, organización, reparto de trabajo y funcionalidades 1, 2 y 3 (base de aplicación y juego). [4 semanas].
    \item Funcionalidades 4 y 5 (sistema de juego offline/online) [4 semanas].
    \item Funcionalidad 6 (realidad aumentada) [4 semanas].
\end{itemize}
%%
\subsection{Responsabilidades}
\begin{itemize}
    \item Óscar Alejandro Manteiga Seoane:
        \begin{enumerate}
            \item Desarrollo de la base de la aplicación (pantallas, ajustes...).
            \item Tablero básico de juego.
            \item Prueba y estudio de ARCore para implementarla una vez llegada su implementación.
        \end{enumerate}
    \item Antonio Vila Leis:
        \begin{enumerate}
            \item Desarrollo del sistema de colocación de barcos en el tablero.
            \item Desarrollo del sistema de juego contra la IA.
            \item Prueba y estudio de ARCore para implementarla una vez llegada su implementación.
        \end{enumerate}
    \item Hugo Sanjiao Varela:
        \begin{enumerate}
            \item Desarrollo de la base de la aplicación (pantallas, ajustes...).
            \item Desarrollo del sistema de juego multijugador.
            \item Redacción de la documentación necesaria.
        \end{enumerate}
    \item Breogán Fernández Moreira:
        \begin{enumerate}
            \item Desarrollo del sistema de juego contra la IA.
            \item Desarrollo del sistema de juego multijugador.
            \item Desarrollo de los test.
        \end{enumerate}
\end{itemize}
Es importante mencionar que la implementación de la realidad aumentada la intentaremos entre todos, ya que ninguno tiene experiencia desarrollándola. Serán Óscar y Antonio los encargados de su estudio a lo largo del desarrollo para poder partir de una base de conocimiento sobre la API ARCore. Las responsabilidades de cada uno pueden variar a lo largo del proyecto.
%%
\subsection{Hitos}
\begin{enumerate}
    \item Entrega de la primera versión de la memoria, con la planificación inicial y requisitos/funcionalidades.
    \item Entrega de la base de la aplicación y del juego.
    \item Entrega del juego completo sin realidad aumentada.
    \item Entrega del juego con la realidad aumentada.
\end{enumerate}
En este tercera entrega, no hemos podido completar el hito marcado, ya que el modo multijugador no funciona completamente. Como le falta poco desarrollo, intentaremos entregarlo junto con la Realidad Aumentada en la última entrega. En caso de que no tengamos tiempo, lo que haremos en lugar de poder jugar en realidad aumentada, será hacer una pequeña demo de su uso.
%%
\subsection{Incidencias}
Posibles incidencias y planes de contingencia:
En caso de problemas genéricos durante el desarrollo, se realizarán pruebas (tests) durante la implementación para intentar corregirlos antes de cada entrega. En caso de encontrarlos, se buscarán alternativas y soluciones en conjunto con todo el equipo. Poseemos los conocimientos necesarios para el desarrollo del juego base, por los que los principales inconvenientes se pueden generar con la implementación de la realidad aumentada, la sincronización online de jugadores y la sincronización de la realidad aumentada.

En el caso de la implementación de la realidad aumentada en el proyecto, si no conseguimos un correcto funcionamiento de esta tecnología, se pasará a desarrollar una aplicación de juegos de mesa online. Los juegos que incorporemos serán varios y distintos entre si para incrementar la complejidad del proyecto.

En caso de poder implementar la realidad aumentada pero sin que sea de forma sincronizada por internet, se creará el juego de hundir la flota de forma local para visualizar tu parte del tablero en esta tecnología mientras que de fondo y para sincronizar el online se juega con el modo básico sin RA.

En el caso de la sincronización online de jugadores, es posible que se produzcan problemas de conexión que afecten a la experiencia de juego de los usuarios. Para minimizar este riesgo, se planearán y realizarán pruebas rigurosas antes de lanzar la aplicación para asegurar la estabilidad del servidor y la capacidad de manejar múltiples conexiones simultáneas.

En cuanto a la sincronización de la realidad aumentada, también pueden surgir problemas técnicos que afecten a la experiencia de los usuarios. Si estos problemas son significativos, se puede optar por desactivar la funcionalidad de realidad aumentada y ofrecer a los usuarios una experiencia de juego sin esta característica.

Otro posible problema es la incompatibilidad de la aplicación con ciertos dispositivos móviles. Para evitar esto, se investigará y se tendrán en cuenta las especificaciones y requisitos de hardware de los dispositivos móviles más utilizados y se realizarán pruebas de compatibilidad antes del lanzamiento.

\section{Observaciones}
En el proceso de desarrollo del proyecto, hemos encontrado ciertas dificultades y desafíos que han afectado nuestra planificación inicial. Algunas de las observaciones y comentarios relevantes son:

\begin{itemize}
  \item En un principio el rendimiento del modo un jugador, multijugador y creación de tablero era completamente nefasto, ya que la creación de tablero se hacía en el xml correspondiente. Finalmente decidimos hacerlo dinámicamente mediante el código en los java mejorando mucho el rendimiento de la aplicación.
  \item Durante la implementación de la funcionalidad de sincronización online de jugadores, nos encontramos con problemas de latencia y retraso en la comunicación, lo que afectó la fluidez del juego. Tuvimos que realizar ajustes y optimizaciones adicionales para mejorar esta parte, sobre todo reduciendo el número de elementos que se envían y reciben.
  \item La implementación de la realidad aumentada resultó ser más compleja de lo esperado. Tuvimos que invertir más tiempo en investigar y comprender la API ARCore, lo que retrasó la integración de esta funcionalidad en el proyecto.
  \item A pesar de nuestros esfuerzos, no pudimos completar el desarrollo del modo multijugador en línea en el tiempo previsto. Decidimos priorizarlo y ofrecer únicamente una pequeña demostración de su uso en lugar de tener un modo de realidad aumentada completamente funcional.
\end{itemize}

Estas observaciones y dificultades nos han enseñado la importancia de adaptar y ajustar nuestras expectativas y planificación durante el desarrollo de un proyecto. Aunque enfrentamos desafíos, pudimos superarlos y lograr resultados satisfactorios en la mayoría de las áreas del proyecto.

%%%%%%%
%%%%%%%
\section{Diseño}
\subsection{Elementos utilizados}
\begin{itemize}
    \item Botones: obviamente, este elemento debe estar presente en cualquier aplicación donde el usuario pueda interaccionar con la misma. Los usamos a lo largo de todas las vistas, pero donde más relevancia tienen son en los tableros. Cuando desarrollamos el código para poder jugar en modo local teníamos los tableros rellenados con un botón por cada casilla en el .xml correspondiente. Esto nos daba muchos problemas de rendimiento e intentamos migrar de los botones a otros elementos sin éxito. Al final, la solución que usamos es la generación de esos botones de los tableros en el .java.
    \item Imagen: la utilizamos para mostrar el logo de la aplicación tanto en la actividad principal como en la de información de la aplicación.
    \item SwitchPreference: para activar o desactivar las notificaciones y activar o desactivar el modo oscuro utilizamos 2 switches en la actividad de los ajustes. Decidimos que era la mejor idea, ya que es un elemento al que el usuario está familizarizado y facilita la implementación.
    \item SeekBarPreference: para controlar el volumen usamos un slider. Consideramos esta la mejor opción ya que el propio Android lo usa también.
    \item ListPreference: en la última entrega incorporaremos traducciones de la aplicación, por lo que el uso de este ListPreference facilitará la elección de idioma que quiera el usuario.
    \item LinearLayout: usaremos esta disposición en la mayoría de actividades de la aplicación por su sencillez y facilidad de organizar los elementos en la pantalla.
    \item RelativeLayout: lo usamos en la actividad de información de la aplicación por su típico uso en estos casos. De esta forma queda todo centrado en la pantalla sin dificultad ninguna.
    \item ScrollView: en la actividad de ayuda usamos una scroll view para que el usuario pueda leer todo el texto haciendo un simple e intuitivo scroll de la pantalla.
    \item TextView: elemento necesario para mostrar texto.
    \item EditText: elemento necesario para que el usuario pueda introducir texto.
\end{itemize}
\subsection{Arquitectura}
La arquitectura de la aplicación se basará en la utilización de diferentes componentes que trabajarán en conjunto para ofrecer una experiencia completa al usuario. Estos componentes serán los siguientes:
\begin{itemize}
    \item Actividades: la aplicación contará con varias actividades que permitirán al usuario acceder a diferentes secciones de la aplicación. Por ahora tenemos las siguientes:
        \begin{enumerate}
            \item Activity de inicio: se ejecutará cuando se inicie la aplicación y mostrará la pantalla de inicio con opciones para iniciar un nuevo juego, cargar un juego guardado (sin confirmar esta funcionalidad) o ver las opciones de configuración.
            \item Activity de creación del tablero: mostrará una pantalla donde el usuario podrá colocar sus barcos sobre su tablero, en esta actividad también se crea el tablero del rival para el modo de un jugador.
            \item Activity de juego "singleplayer": mostrará el tablero del jugador y la IA, permitiendo jugar contra ella.
            \item Activity de juego "multiplayer": mostrará el tablero de jugador y el del oponente, permitiendo jugar contra el.
            \item Activity de Realidad Aumentada: mostrará una pequeña demostración de uso de la Realidad Aumentada.
            \item Activity de configuración: permitirá a los usuarios personalizar la configuración de la aplicación, como cambiar el idioma, la configuración de sonido, las notificaciones y el modo oscuro.
            \item Activity de ayuda: mostrará información de ayuda y tutorial para los nuevos usuarios.
            \item Activity de información: proporcionará información de la aplicación y de los autores de la misma.
            \item Activity de lobby: mostrará una serie de salas a las que el usuario puede unirse para conectarse a un juego multijugador, así como la posibilidad de crear una nueva sala.
            \item Activity de fin del juego: mostrará el resultado de la partida (si se ha ganado o perdido) y un botón para regresar a la activity de inicio.
        \end{enumerate}
    \item Servicios: para poder jugar online, usamos Firebase Realtime Database como servicio para sincronizar los arrays de los tableros de ambos jugadores, las salas de juego y el almacenamiento de datos que puedan ser útiles en un futuro (nombre de jugador, resultado de la partida...). Para acceder a esto, fue necesario registrarse en Firebase, crear una aplicación en su consola, descargar unos ajustes para poder funcionar con la aplicación de Android Studio y crear una Realtime Database donde guardar los datos.
\end{itemize}

\subsection{Persistencia}
Como mencionamos en el punto anterior, la persistencia de datos online la hacemos con Firebase Realtime Database. En cuanto a los datos en local (partidas contra la máquina, preferencias...) usamos el almacenamineto del dispositivo de cada usuario.

\subsection{Vista}
La aplicación contará con varias actividades y fragmentos que permitirán al usuario acceder a diferentes secciones de la aplicación. Además, se incluirán notificaciones en el último entregable. A continuación, se describen los elementos visuales que se incluirán en la aplicación:
\begin{itemize}
    \item Actividades: la aplicación contará con varias actividades, como la creación del tablero, la ejecución de las partidas, el menú principal y las pantallas de ayuda o configuración.
    \item Diálogos: se utilizarán diálogos para que el usuario pueda ingresar el nombre de una nueva sala que se disponga a crear o para unirse a una existente y poder ingresar su nombre.
    \item Menús: casi la totalidad de las actividades tendrán un menú en la parte superior para acceder a los ajustes, ayuda o información sobre la aplicación. A mayores, si no estamos en la actividad principal, también tendremos una flecha para retroceder a la actividad anterior.
\end{itemize}

\subsection{Comunicaciones}
Las comunicaciones con el servidor de Firebase se hará a través del enlace que nos proporciona esta herramienta para poder acceder a la base de datos. No se puedo por falta de tiempo realizar una autenticación de usuarios e incremento de la seguridad, ya que actualmente la base de datos está abierta a posibles borrados de información.

\subsection{Sensores}
ARCore proporciona acceso a diversos sensores del dispositivo que son fundamentales para la realidad aumentada. Algunos de los sensores más comúnmente utilizados en la API ARCore son:

\begin{itemize}
  \item Sensor de movimiento: ARCore utiliza el acelerómetro y el giroscopio del dispositivo para detectar y rastrear los movimientos y cambios de orientación.
  
  \item Sensor de luz ambiental: Este sensor permite a ARCore medir la cantidad de luz ambiental en el entorno, lo que ayuda a que los objetos virtuales se integren mejor con el mundo real en términos de iluminación.
  
  \item Sensor de profundidad: Algunos dispositivos admiten sensores de profundidad, que permiten a ARCore capturar información tridimensional sobre la escena. Esto es especialmente útil para la detección precisa de planos y la colocación de objetos virtuales en el entorno.
  
  \item Sensor de cámara: La cámara del dispositivo se utiliza para capturar imágenes y videos del entorno. ARCore utiliza estas imágenes para realizar el seguimiento y la detección de características visuales, lo que permite superponer objetos virtuales de manera precisa.
\end{itemize}

Estos sensores son esenciales para brindar una experiencia de realidad aumentada fluida y precisa. ARCore se encarga de gestionar y utilizar estos sensores de manera eficiente para ofrecer un rendimiento óptimo en las aplicaciones de realidad aumentada.

\subsection{Trabajo en background}
Las tareas en segundo plano en nuestra aplicación son relativas a la actividad de multijugador. La primera sería un Listener que se mantiene escuchando cualquier cambio en la base de datos, para así poder actualizar los tableros de los jugadores. La segunda serían los dos temporizadores con los8 que cuenta cada jugador: el que realiza la cuenta atrás del mismo y el que se asegura que el temporizador del rival se actualice (detectando así posibles caídas en la conexión).

\section{Pruebas y Resultados finales}
\subsection{Pruebas}
Se han realizado numerosas pruebas durante el desarrollo y también varias pruebas una vez este fue finalizado y se ha comprobado que la aplicación funciona correctamente en todos los casos. Pueden ocurrir que en los ajustes no corresponda el switch del tema oscuro y el selector de idioma con el del dispositivo, pero con seleccionarlo una vez ya quedará bien establecido. Esto sucede por tenerlos puesto de forma manual, ya que no lo automatizamos (aunque si está el código), para comprobar más fácilmente si funcionan estas características.

\subsection{Resultados Finales}
Se ha conseguido hacer una aplicación multijugador online y con modo de un jugador contra la IA, incorporando también una prueba de realidad aumentada. No obstante, no se ha conseguido incorporar la RA al bucle de juego, elemento que dejaremos como trabajo futuro.

\section{Conclusiones}
Hemos conseguido realizar una aplicación basada en el juego de hundir la flota, con sus funcionalidades originales (crear tableros, atacar casillas, etc), y con la posibilidad de jugar contra la IA (singleplayer), o contra otros jugadores (multiplayer). Para el modo multijugador utilizamos una base de datos Firebase para el control de datos. Si bien la idea original era hacer que el juego se desarrollase completamente en realidad aumentada, nos encontramos con que la incorporación de esta tecnología es muy compleja. Aún así, hemos implementado una prueba de RA, en la que se puede ver el modelo de un barco.

\section{Trabajo Futuro}
En un futuro se buscará integrar la Realidad Aumentada al ciclo del juego, de forma que mediante esta se pueda representar el estado de los tableros de los jugadores. A mayores, podríamos mejorar la estética del juego añadiendo iconos de barcos en lugar de casillas de colores, personalización de los barcos, puntuaciones e incluso multijugador competitivo.

\begin{enumerate}
    \item Integración de Realidad Aumentada (RA): La integración de la Realidad Aumentada podría permitir a los jugadores ver visualmente el estado de los tableros de juego en sus entornos físicos. Por ejemplo, podrían colocar sus dispositivos móviles o utilizar dispositivos de RA especiales para ver los barcos, las casillas y los efectos visuales del juego superpuestos en tiempo real en su entorno real. Esto añadiría una capa adicional de inmersión y realismo al juego de Batalla Naval.
    \item Mejoras estéticas: En lugar de las casillas de colores, se podrían utilizar iconos de barcos para representar las posiciones de los barcos en los tableros de juego. Esto permitiría una representación visual más intuitiva y atractiva del juego. Además, se podrían añadir efectos visuales y sonoros para destacar los impactos de los disparos y los hundimientos de los barcos.

    \item Personalización de los barcos: Para añadir un elemento de personalización, se podría permitir a los jugadores personalizar sus propios barcos. Esto podría incluir opciones para cambiar el diseño, el color, la forma y otros aspectos visuales de los barcos. También se podrían desbloquear diseños especiales o skins a medida que los jugadores progresen en el juego.
    
    \item Puntuaciones y clasificación: Se podría implementar un sistema de puntuación para que los jugadores compitan por obtener la puntuación más alta en cada partida. Esto podría basarse en criterios como el número de barcos hundidos, el tiempo empleado en completar la partida o la precisión de los disparos. Además, se podría incluir una clasificación global para que los jugadores puedan comparar sus puntuaciones con las de otros jugadores de todo el mundo.
    
    \item Multijugador competitivo: Una adición emocionante sería la introducción de un modo multijugador competitivo en línea. Esto permitiría a los jugadores desafiar a sus amigos o a otros jugadores en partidas en tiempo real. Se podrían incluir características como la creación de salas personalizadas, el emparejamiento automático de jugadores de habilidades similares y tablas de clasificación para mostrar los mejores jugadores.
    
    \item Ampliación de la variedad de modos de juego: Además del modo clásico de Batalla Naval, se podrían añadir nuevos modos de juego para mantener el interés de los jugadores. Algunas ideas podrían ser modos de juego por equipos, modos de juego por tiempo, modos de juego con habilidades especiales o incluso modos de juego cooperativos donde los jugadores se unen para enfrentarse a desafiantes enemigos controlados por la inteligencia artificial.
    
\end{enumerate}

\newpage

\bibliographystyle{pfc-fic}
\bibliography{biblio}
\addcontentsline{toc}{section}{Bibliografía}

\end{document}
